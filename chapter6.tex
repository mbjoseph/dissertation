\chapter{Host diversity begets symbiont diversity but reduces symbiont transmission}
\chapterauthor{Maxwell B. Joseph, Joseph R. Mihaljevic, Pieter T. J. Johnson}

Explaining patterns of biodiversity is one of the primary aims of ecology, and historically these patterns have been described in terms of macroorganisms such as plants and animals.
Recent methodological advances grant the ability to observe previously unobservable microbial communities, which are incredibly diverse and often nested within free-living organisms, providing an impetus and means to expand understanding of the mechanisms that structure communities.
These mechanisms are potentially similar to the mechanisms that structure free-living communities, with habitat patches for free-living species being analogous to host individuals for symbiotic organisms, often microbial, that live in or on free-living species.
Here, we apply this analogy to explore the possible effects of host functional diversity on symbiotic organisms with a stochastic multi-species model across a range of scenarios including specialist, generalist, parasitic, commensal, and mutualistic symbionts.
The model predicts that host diversity nearly always increases symbiont diversity, though the shape of the increase varies as a function of symbiont niche breadth and the fitness consequences of infection.
In addition, diverse host communities are predicted to have lower transmission rates on average, and relatively greater transmission of generalist symbionts.
The same mechanisms allow host diversity to increase symbiont diversity while reducing transmission.
Namely, diverse host communities have higher symbiont colonization rates but fewer hosts available for local transmission, which increases the probability of stochastic symbiont extirpation.
These results are potentially useful as a step toward a more unified community ecology of free-living and symbiotic organisms.

\section{Introduction}

Recent progress in disease ecology has highlighted the importance of host diversity for infectious disease dynamics, but relatively little is known about the role of host diversity in regulating the dynamics of transmissible symbionts in general \citep{Lively2014, Johnson2015}.
Specifically, the joint influence of host diversity on symbiont transmission and diversity remains poorly understood.
Theory has been developed to understand how host community composition affects the transmission dynamics of symbiotic pathogens \citep{Dobson2004, Roche2012, Joseph2013a, Mihaljevic2014}, and biogeographic studies indicate positive correlations between host and parasite diversity \citep{Nunn2005, Dunn2010a, Harris2010}.
Methodological advances may now facilitate joint consideration of these phenomena, as next generation sequencing technologies increasingly reveal community composition of symbionts within host populations and communities \citep{Prosser2007,Fierer2012,Costello2012a}.
Despite this opportunity to observe symbiont communities, we currently lack theoretical predictions about host community composition influence on symbiont diversity and transmission dynamics, possibly due to computational challenges in modeling these complex systems across multiple scales \citep{Gog2014}.
Here, we make an effort to overcome these challenges, presenting a general model that explores how host biodiversity affects the transmission and biodiversity of parasites, mutualists, and commensal symbionts.

To predict how host diversity generally influences symbiont diversity and transmission, we may be able to apply ecological and biogeographical concepts developed for free-living communities to symbiont communities inhabiting multiple host species.
For instance, analogies have been drawn between free-living communities living within habitat patches and symbiont communities living inside host individuals \citep{Mihaljevic2012,Richgels2013,Suzan2015}.
One way to view host biodiversity from the symbionts' perspective is through the concept of habitat heterogeniety, where individuals from different host species present unique environmental conditions that could influence the population dynamics of symbionts.
Given this analogy, we can then test whether predictions from free-living systems hold for symbiont communities, which have unique dynamics compared to free-living species.

In free-living systems habitat heterogeneity can increase species coexistence locally and regionally, because diverse habitats contain resources that can be exploited by multiple species and source-sink dynamics are more likely to rescue species from local extirpation \citep{Levins1969,Pacala1982,Amarasekare2004}.
For a habitat region of fixed area, as habitat heterogeneity increases, the area of each habitat type must go down because of an inherent trade-off with habitat heterogeneity.
As a result, mean population sizes have been predicted to decrease with habitat heterogeneity, leading to higher probabilities of stochastic extinctions \citep{Allouche2012}.
Thus, when habitat heterogeneity increases, but habitat area remains fixed, a unimodal relationship between heterogeneity and free-living species richness is expected due to low persistence probabilities at very high habitat heterogeneity \citep{Allouche2012}.
However, it remains an open question whether this ecological theory pertaining to habitat heterogeneity is relevant to explaining patterns of symbiont diversity and transmission.

In this study, we use a computational model to explore whether host diversity influences symbiont communities in a way analogous to the effects of habitat heterogeneity in free-living systems.
We do this by simulating communities of many host and symbiont species and manipulating host functional diversity and characteristics of the symbiont community.
Our model is unique in that it incorporates ecological dynamics important for free-living host population dynamics, such as colonization and reproduction, as well as mechanisms exclusive to host-symbiont interactions, such as transmission and symbiont-mediated fitness effects on the host.
We use this modeling framework to determine how host functional diversity could affect symbiont transmission and symbiont species richness over a range of scenarios.
Specifically, in our model we can manipulate symbiont niche breadths - allowing for a combination of specialist and generalist symbionts - and symbiont-mediated fitness effects - allowing for a combination of parasites, commensals, and mutualists.
This allows simultaneous investigation of the effect of host biodiversity on local symbiont transmission and diversity.

\section{Methods}

We developed a stochastic, continuous-time, agent-based model for the dynamics of host-symbiont communities.
In the model, a host community resides in a habitat patch that is divided into a finite number of cells $C$, i.e. micro-sites which can be occupied by a single host individual.
Within the habitat patch, host colonization, reproduction, and death occur with rates $c$, $r$, and $d$, respectively, which are constant across model realizations.
Multiple host species $h=1, ..., H$ exist within a regional pool, and the patch experiences a constant rate of host colonization $c$, which is equal for all host species, and there is no displacement of occupying hosts by colonizing hosts.
Host species vary in traits that are relevant to symbiont infection, but the communities are otherwise neutral, in that all host species have equal colonization, reproduction, and death rates.
Once a host colonizes a cell, it then has some rate of reproduction and of death that is constant among host species.
Offspring attempt to disperse to a random habitat patch cell, and if it is unnoccupied, they successfully colonize.

Multiple types (species) of symbionts occur in the regional pool and attempt to colonize host individuals with a constant rate $r_s$.
Host species vary in traits relevant to symbiont establishment probability.
We assume for simplicity that this variation is unidimensional, as would be the case if there were one primary niche axis for the symbionts, e.g. pH of the host gut.
Symbiont niches are represented as univariate Gaussian functions with niche optima and breadths (means and variances, respectively), such that the probability of successful establishment $p_e$ on an individual host is a function of the host's value along the niche axis (Figure \ref{fig:niche}).
Symbionts have the potential to establish when a susceptible host contacts an infected host, or when a symbiont from the regional pool attempts to colonize a susceptible host in the habitat patch.

Every individual host has some constant probability of a symbiont from the regional pool attempting to infect, with an infection establishment probability $p_e$ from the symbiont niche function.
If a symbiont successfully colonizes a host, it becomes part of the local community and can either be extirpated through host recovery or persist by transmitting to other hosts.
Every infected host has a recovery rate that is assumed constant across all species.
Hosts contact each other with rate $\phi$, also assumed constant across host species, so that the overall rate of contacts between susceptible and infected individuals is $\phi S I$, where $S$ is the number of susceptible hosts, and $I$ is the number of infected hosts.
Conditional on a contact, the infected host passes the symbiont infection to the susceptible host probability $p_e$, derived from the symbiont niche function.
In this way, host species that are functionally similar are more likely to exchange symbionts.

We implemented this model via stochastic simulation, generating continuous-time Markov chains in high-dimensional space, where the dimensions represent the states: number of hosts of each species, and the symbionts that infect each individual.
Although continuous-time Markov processes are more difficult computationally than discrete-time formulations, they have the advantage of not requiring an arbitrarily specified order of events at each time step.
The order of events emerges from the rates of each potential process via the Gillespie algorithm \citep{Gillespie1976}.
Given the state of the system at any time point, every potential event that could happen is represented by some rate $r_e$ for events $e = 1, ..., E$.
The total rate of events is simply the sum of the component rates $r_{tot} = \sum_{1}^{E} r_e$, and the time until an even happens is exponentially distributed, with overall rate parameter $r_{tot}$.
This provides a way to generate the timepoints stochastically, and given some time of an event happening, the specific event is be selected with probability $e_r / r_{tot}$.

We were interested broadly in the effect of host functional diversity on symbiont transmission and richness within a local host community.
To vary host functional diversity, we specified a range of within-host environmental conditions and uniformly simulated values bounded within this range for each host species, and assigned these values to host species in the regional pool.
The realized functional diversity within a local community is the range of values represented by host species that occupy the local community, and can thus vary across a simulated time series as new host species colonize or are extirpated.
We monitored the number of symbiont transmission events and symbiont species richness throughout each simulated time series.
In order to compare values across many simulated time series, we restrict our calculations of mean host functional diversity, symbiont transmission, etc. to common periods in which the quantities have stabilized, rather than including the entire time series.
While the contact rate $\phi$ is a model parameter, the transmission rate for any time period is a model outcome, calculated as the number of transmission events divided by the amount of time.

In addition to investigating the effect of host functional diversity on symbiont transmission and richness, we explored the effect of symbiont niche breadth ($\sigma$ in Figure \ref{fig:niche}), and the effect of infection on host fitness, described below.
These questions lead to six different simulation experiments. %It might be worth ordering these a bit more logically. For instance, the second and fourth experiments seem related, (and could therefore be second and third) and same with the third and fifth.
The first explores the effect of host functional diversity on transmission and symbiont richness when all symbionts are functionally equivalent and commensal.
The second allows symbiont niche width to vary among simulated realizations, but assumes that the niche width of all symbionts is equal within a simulation run.
The third varies the effect of symbiont infection on host survival, varying across simulation runs but equal among symbionts within a run.
The fourth varies symbiont niche width within and across simulation runs so that the regional pool of symbionts includes specialists and generalists.
The fifth varies symbiont fitness effects within and across runs so that the regional pool includes parasites and mutualists.
The sixth and final experiment allows both niche width and fitness effects to vary within and across runs so that the regional pool of symbionts includes a range of specialist and generalist parasites and mutualists.
All code to replicate the analysis and manuscript is publicly available at \url{https://github.com/mbjoseph/abm}.

\section{Results}

In our first experiment all symbionts were commensal and had equal niche widths.
Across 1000 simulated local host communities, host functional diversity increased symbiont richness and decreased transmission (Figure \ref{f2}).
Symbiont diversity increased non-linearly with host functional diversity, with the fastest increases occurring at low levels of host functional diversity.
While symbiont transmission was lower on average in communities with high symbiont diversity and high host functional diversity, there was one instance in a low-diversity host community in which no symbionts could colonize, because the range of host resources that were available in the local community was outside the optimal niche space of the symbionts in the regional pool.
In other words, symbionts attempting to colonize had very low probabilities of establishment in the hosts, such that across the range of timesteps, no symbionts succeeded in colonizing the local community.

In our second experiment, where the niche width of symbionts varied among simulations but was constant among species within each simulation, host diversity still increased symbiont richness and decreased transmission, and these relationships were strongly moderated by symbiont niche width (Figure \ref{f3}).
The increases in symbiont richness with host richness were still non-linear, but they shifted from being concave down to concave up as symbionts became more generalized.
Transmission consistently decreased as host communities became more functionally diverse, but as symbiont niche width increased, there were fewer outlier communities with extremely low transmission, because it becomes less likely that any particular host community will have a range of resources available that cannot be exploited by symbionts in the regional pool because of a mismatch with their niche space.

In our third experiment, where the fitness consequences of infection varied among simulations, host diversity still increased symbiont richness and reduced transmission, but parasites had lower richness and transmission on average compared to mutualist and commensal symbionts (Figure \ref{f4}).
Parasites restrict their own transmission by killing hosts, making it more difficult to persist in a local host community \citep{Wood1996}.
As a result, richer host communities still can be colonized by more parasitic symbiont species, but these communities will have more depauperate symbiont communities relative to communities comprised of less virulent symbionts.

When symbiont niche width varied within communities, host diversity still increased symbiont richness, but the effects of host diversity on symbiont transmission were more subtle (Figure \ref{f5}).
Specialist symbionts had the highest transmission rates in low-diversity communities, but had relatively low transmission in more diverse host communities.
Generalist symbionts had higher transmission in the high diversity host communities, and lower transmission in less diverse host communities, consistent with the idea that specialist symbionts can outcompete generalists.
In addition, mean niche breadth in the symbiont community increased with host functional diversity, such that diverse host communities contained more generalist symbionts than low diversity host communities.

When symbionts varied in fitness effects, host diversity increased symbiont richness, and reduced transmission, but the reduction in transmission was more pronounced for parasitic than mutualistic symbionts (Figure \ref{f6}).
Interestingly, extremely mutualistic symbionts had a lower peak transmission in low diversity communities than moderately mutualistic symbionts, because hosts were so long-lived that the majority of habitat patches in the lowest diversity communities were occupied by extremely long-lived, infected hosts, such that there were fewer susceptible hosts available for transmission.

When symbionts varied in fitness effects and niche breadth, host diversity still increased symbiont diversity and reduced transmission (Figure \ref{f7}).
Mutualistic symbionts nearly always had higher transmission than parasitic symbionts, and on average, symbionts in the local communities were mutualistic.
Furthermore, low diversity host communities had more specialist symbionts than high diversity communities, which primarily consisted of generalists.
Both of these findings are consistent with the component cases when fitness effects and niche breadth are the only varying parameters.

\section{Discussion}

Across a wide range of situations, increasing host functional diversity consistently increased symbiont diversity and reduced average transmission in our model.
Fitness consequences of infection and symbiont niche breadth regulated the degree and, to a lesser extent, the shape of these relationships.
In general, we expect that across replicated communities in nature, with all else being equal, host communities that are low in functional diversity will have relatively low symbiont richness but high average transmission per symbiont species, and that specialist symbionts may be more likely to occur in high abundance.
In contrast, host communities that are highly diverse are more likely to have diverse symbiont communities with lower mean transmission rates, and also more generalist symbionts.

Recent theory has clarified conditions under which host biodiversity might affect transmission of parasitic symbionts, but relatively few studies have considered multiple symbiont species simultaneously.
For instance, we know that host community richness, functional diversity, evenness and the distribution of life-history traits can all affect the transmission of a focal pathogen \citep{Dobson2004, Rudolf2005, Roche2012, Joseph2013a, Mihaljevic2014, Chen2015, ORegan2015a}.
Here we have generalized previous theory to include multi-host, multi-symbiont dynamics, allowing us to consider the effects of host biodiversity on patterns of symbiont diversity and transmission simultaneously.
Further, we have expanded previous models to allow for symbionts with variable effects on host fitness, including parasitic, mutualistic and commensal organisms, which more closely matches the composition of natural symbiont communities.
Under a broad set of assumptions, host diversity promoted more diverse symbiont communities, and symbiont diversity scaled linearly, concave up, and concave down with host diversity.
All of these functional forms have been predicted by analytical models for parasites in host communities undergoing disassembly and have been observed in some host-parasite food webs \cite{Lafferty2012}.
In addition, consistent with previous models that vary host diversity and in which host density does not systematically increase host diversity \citep{Mihaljevic2014}, we found that transmission tends to be reduced in diverse host communities \citep{Dobson2004, Rudolf2005, Roche2012, Joseph2013a, Mihaljevic2014}.

Here we have shown that these general predictions may hold not only for parasites, but also for commensal and perhaps mutualistic organisms, and that the same mechanisms underlie both patterns of increased richness and reduced transmission.
Specifically, in low diversity host communities, all hosts are functionally similar so that they readily share symbionts upon contact - leading to high transmission.
Yet, the pool of potential symbiont colonists is small, restricted to the subset of symbionts that can effectively exploit the narrow range of host resources that are represented in the local community - leading to low symbiont diversity.
In high diversity host communities, different host species generally have unique resources available for symbionts, so that when any two individual hosts of randomly chosen species make contact, they are less to transmit their symbionts.
In addition, because we have fixed community carrying capacity, increases in host diversity imply that there are fewer individuals of each host species present.
Thus, ther are fewer host 'habitats' that represent any particular set of conditions for symbionts, resulting in a decrease in transmission.
Simultaneously, the number of symbionts in the regional pool that can establish in the local community is greater in diverse communities, where a broader set of resources are represented.
So, while more symbiont species may colonize the local community, on average their transmission is reduced.
These results have strong connections to theory developed around relationships between habitat heterogeneity and free living diversity.

\cite{Allouche2012} predicted that for a fixed habitat area, increases in habitat diversity may lead to concave down and even unimodal relationships between habitat heterogeneity and free-living richness, and the mechanisms underlying this pattern are analogous to the mechanisms discussed in this paper.
Hosts have long been compared to habitat patches for symbionts, except these habitat patches are generally mobile, reproduce, compete with and contact each other, and experience differential survival and reproduction as a result of symbiont occupation or infection \citep{Kuris1980, Mihaljevic2012}.
In free living species, for fixed area, increasing habitat diversity tends to reduce the mean area of each habitat type.
Similarly, for a fixed number of host individuals, increases in host diversity reduces the number of any particular host.
Increases in habitat or host diversity increase the colonization rate within local communities, while decreases in the abundance of a habitat or host type increase the probability of stochastic extinctions for free-living and symbiotic species.
While we have not included the hosts' habitat heterogeneity in this model, it would be a logical next step to explore how this habitat heterogeneity could jointly influence host and symbiont diversity.
This type of modeling effort, however, would require more assumptions about the relationship between host and symbiont niches, which unfortunately are currently ill-informed by empirical data.

As it has been currently formulated, the predictions from this model are broadly consistent with existing empirical data on the apparent effects of host diversity on symbiont transmission and diversity.
Specifically, on average diverse host communities have lower abundances of parasitic symbionts \citep{Civitello2015}, although this claim and topic in general have been contentious \citep{Salkeld2013,Lafferty2012}.
While parasite abundance is not a direct measure of transmission, theoretical models predict that the reduction of transmission is a likely mechanism underlying the abundance pattern \citep{Dobson2004, Rudolf2005, Mihaljevic2014}. %It's unclear to which abundance pattern you're referring
Here we have broadened the focus to include commensal and mutualistic symbionts, and this model's expectations for the transmission of parasites are not vastly different than the predictions for transmission of mutualists and symbionts.
Broadening the focus of the diversity-disease literature to include mutualists and symbionts is more important and feasible now more than ever, given our capacity to collect massive amounts of data on microbes using new sequencing methods. %e.g. Hamady & Knight 2009_GenomeRes, Kuczynski et al. 2012_NatRevGen
Aside from transmission, it has long been recognized that host diversity facilitates parasite diversity, and this model provides a basis for this relation along with more broadly applicable predictions for specialist vs. generalist and parasitic vs. mutualistic symbionts. %citations: Hetchinger_2006?, Dunn et al 2010, Harris and Dunn 2010

While this model represents a step toward more realistic representations of the dynamics of multi-host, multi-symbiont systems, as with any model, we have made simplifying assumptions that could be relaxed in future studies.
Specifically, by including many species of hosts and symbionts, our model has many parameters, and with a simulation approach we can only explore a subset of parameter space.
While an analytic model would be useful, the mathematical complexity of such models increases rapidly as species are added.
In addition, we assumed that hosts could only be singly infected - that they could not be infected with multiple symbiont species simultaneously.
We know that co-infection is common in nature, but including co-infection in the model would require additional sets of assumptions about interactions between symbionts within host individuals (e.g., can symbionts displace each other, is there facilitation among symbionts, competition, neutral interactions, etc.), reducing the tractability of the model. %citations: Petney and Andrews 1998_IntJParasit, Cox 2001_Parasit, Telfer et al 2010_Science, Griffiths et al 2011_JInfection
In addition, we also know that local habitat patches are often connected, raising questions about the metacommunity dynamics of the hosts and symbionts, which can be considered to exist within their own metacommunities among host individuals within a community \citep{Mihaljevic2012}, however in this model we did not explicitly model metacommunity dynamics among habitat patches. %also Suzån et al 2015_Ecology and Evolution
Last, we have assumed direct horizontal transmission for all symbionts, but in nature, many other forms of transmission are common, particularly for mutualistic symbionts which are commonly vertically transmitted.

Beyond these ecological phenomena that were not included, we also did not include evolutionary change in either the symbionts or the hosts.
However, this represents a key extension to consider in the future, as the confluence of ecological and evolutionary timescales is one of the hallmarks of microbe-host dynamics.
There are many issues to consider from an evolutionary perspective for these types of models, including the evolutionary pressures imposed by the assembly and disassembly of host communities, and the selective effects of habitat or host diversity on symbiont traits.
Finally, while this model may be useful in generating qualitative predictions, because it is not parameterized for any particular system or set of species, validation may be more difficult.
The predictions from this model may be best considered to be broad expectations for patterns seen across many replicated communities.

Taken together, the predictions for this model serve to unite two previously separate ideas, that diverse host communities may have more diverse symbiont communities and that diverse host communities can lead to lower transmission of a focal symbiont.
Our perspective, which includes parasites but also commensals and mutualists, is more general and probably less prone to vitriol than the perspective under scrutiny in the diversity-disease debate, which is focused on the somewhat vaguely defined outcome of disease risk \citep{Ostfeld2003}.
Possibly, a consideration of symbiont diversity and transmission as an outcome is more precise.
There are many additional questions to be explored regarding the relationship between host and symbiont diversity, and we have attempted to outline these above.
These theoretical questions have empirical counterparts for future work in this arena which has the potential to leverage new technologies in entire community sequencing to rapidly acquire data.

\begin{figure}[ht]\centering
\includegraphics[width=\linewidth]{figs/ch6/niche.pdf}
\caption[Niche space for symbionts]{Conceptual diagram illustrating the niche space of symbionts along a univariate host-condition axis (the x-axis). Each host species in a local host community is assigned one particular value, shown as vertical bars. The range of the values represents the host functional diversity represented in the local community. Here two symbiont species have equal niche breadths $\sigma$ but different optima $\mu$, which together define the Gaussian function that relates host condition to the probability of symbiont establishment, conditional on an infection attempt.}
\label{fig:niche}
\end{figure}

\begin{figure}[ht]\centering
\includegraphics[width=\linewidth]{figs/ch6/fig1.pdf}
\caption[Predictions when niche widths are equal]{Model predictions for commensal symbionts that all have the same niche width. Panel A shows the relationship between mean host functional diversity and symbiont richness, and panel B shows the relationship between host functional diversity and symbiont transmission rate. Here, $C=500$, $H=50$, $S=50$, $\sigma = 1$, $c=0.0001$, $r=0.1$, $d=0.08$, $r_s=1$, and $\phi = 10$.}
\label{f2}
\end{figure}

\begin{figure}[ht]\centering
\includegraphics[width=\linewidth]{figs/ch6/fig2.pdf}
\caption[Predictions when niche widths vary among communities]{Model predictions with varying symbiont niche widths among time series (though constant within). Symbiont niche widths were uniformly generated between $\sigma_{min} = 0.5$ and $\sigma_{max} = 50$, and as before $C=500$, $H=50$, $S=50$, $c=0.0001$, $r=0.1$, $d=0.08$, $r_s=1$, and $\phi = 10$. Colors represent symbiont niche width and encode the same information as the panels, but aid in comparisons between the symbiont richness plots (panel A) and the symbiont transmission plots (panel B).}
\label{f3}
\end{figure}

\begin{figure}[ht]\centering
\includegraphics[width=\linewidth]{figs/ch6/fig3.pdf}
\caption[Predictions when fitness consequences vary among communities]{Model predictions with varying fitness consequences of infection among time series. Symbiont fitness effects $\beta_d$ were uniformly generated between $-0.99$ and $10$. Positive values of $\beta_d$ imply increased death rates with infection and more parasitic symbionts, while negative values of $\beta_d$ imply mutualistic symbionts that reduce host death rates with infection. Colors represent the particular values of $\beta_d$ selected for each simulated time series. Here, $C=500$, $H=50$, $S=50$, $\sigma = 1$, $c=0.0001$, $r=0.1$, $d=0.08$, $r_s=1$, and $\phi = 10$.}
\label{f4}
\end{figure}

\begin{figure}
\centering
\includegraphics[width=\textwidth,height=\dimexpr\textheight-4\baselineskip-\abovecaptionskip-\belowcaptionskip\relax,
keepaspectratio]{figs/ch6/fig4.pdf}
\caption[Predictions when niche breadth varies within communities]{Model predictions for mixed niche breadths within simulated time series. Here, symbiont niche breadths vary among symbiont species and are uniformly generated on the interval $(0.5, 50)$. Panel A shows the scaling of host functional diversity with symbiont richness, with the mean symbiont niche breadth in the local community shown in color. Panel B shows the transmission rates for each particular symbiont species by time series combination with panels corresponding to niche width bins. Here, $C=500$, $H=50$, $S=50$, $\sigma = 1$, $c=0.0001$, $r=0.1$, $d=0.08$, $r_s=1$, and $\phi = 10$.}
\label{f5}
\end{figure}

\begin{figure}[ht]\centering
\includegraphics[width=\textwidth,height=\dimexpr\textheight-4\baselineskip-\abovecaptionskip-\belowcaptionskip\relax,
keepaspectratio]{figs/ch6/fig5.pdf}
\caption[Predictions when fitness consequences vary within communities]{Model predictions for mixed fitness consequences within simulated time series. Symbiont effects on the host death rate vary among symbiont species and are uniformly generated on the interval $(-0.99, 0.99)$. Panel A shows the scaling of host functional diversity with symbiont richness, and mean fitness effects are shown in color. Panel B corresponds to the transmission rates for each symbiont species in each time series with panels corresponding to death rate effect bins. If the death rate effect is negative, then infected hosts have lower death rates, surviving longer on average. If positive, hosts die faster. Here, $C=500$, $H=50$, $S=50$, $\sigma = 1$, $c=0.0001$, $r=0.1$, $d=0.08$, $r_s=1$, and $\phi = 10$.}
\label{f6}
\end{figure}

\begin{figure}[ht]\centering
\includegraphics[width=\textwidth,height=\dimexpr\textheight-4\baselineskip-\abovecaptionskip-\belowcaptionskip\relax,
keepaspectratio]{figs/ch6/fig6.pdf}
\caption[Predictions when communities include symbionts with varying niche widths and fitness consequences]{Model predictions for mixed symbiont fitness consequences and niche breadths within simulated time series. Symbiont effects on the host death rate vary among symbiont species and are uniformly generated on the interval $(-0.99, 0.99)$, and symbiont niche breadths vary among symbiont species and are uniformly generated on the interval $(0.5, 50)$. Panel A shows the scaling of host functional diversity with symbiont richness. Panel B and C show the relationships between mean mortality effects and niche breadths with host functional diversity. Panel D corresponds to the transmission rates for each symbiont species in each time series with panels corresponding to death rate effect and niche breadth bins. Here, $C=500$, $H=50$, $S=50$, $\sigma = 1$, $c=0.0001$, $r=0.1$, $d=0.08$, $r_s=1$, and $\phi = 10$.}
\label{f7}
\end{figure}
